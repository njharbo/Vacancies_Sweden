\section{Conclusion}
\label{sec:conclusion}

In modern macroeconomic models job-openings are a key concept. To measure job-openings in the data the literature relies on vacancy data. Such data is either measured in surveys or in register data. Yet so far we know little about how job-vacancies relate to actual hiring on the micro level. Insofar that job vacancies capture the concept of job-openings well, we should expect to see a tight relationship between vacancies and subsequent hires.
 
This paper is among the first to study this relationship on the plant level. Using a new dataset, I show that the relationship between job-vacancies and subsequent hires is weak and concave. That is, one additional vacancy on the plant level predicts less and less hiring. I also show that the number of hires per vacancies (the \emph{vacancy yield}) varies in the cross-section of plants. In particular, it is increasing in plant size. %Moreover, plants hire many more workers than they post vacancies. I take these findings as evidence of vacancies being a poor measure of actual job-openings in the economy. 

These plant-level findings have at least two potential interpretations. 

One interpretation is that the traditional vacancy measure does a poor job in measuring actual job-openings. Building on this interpretation, I build an alternative measure for job-openings. Specifically, I use the \emph{number of plants with a positive number of vacancies weighted by their size}. This measure is motivated by the concave relationship between vacancies and hires and the predictive power of plant size. I show that using this measure yields a better fitting matching function on the aggregate level. This interpretation can partly explain why the job-finding probability is lower than what a matching function estimated on historical data predicts. Specifically, the predicted job-finding probability during 2008-2013 is on average 2.2 percentage points lower than when using the traditional vacancy measure. This constitutes 24 \% of the post Great Recession breakdown in the relationship between hires and labor market tightness. 

Another interpretation is that the \emph{match efficiency} of vacancies varies in the cross-section of plants. To investigate the implication of this I write down a simple matching model with heterogeneity in the match efficiency across plant-size. Here a shift in the distribution of vacancies towards plants with lower match efficiency will lower the job-finding probabilities. In this model, the compositional change of vacancies during 2008-2013 can however only explain very little of the fall in the job-finding probability. Consequently, this model fails to explain the post Great Recession breakdown in the relationship between hires and labor market tightness.

Overall, my findings suggest that more work is needed on how to best measure job-openings in the economy. A substantial amount of hiring happens without preceding vacancies. This points to a reliability problem in our vacancy data. Understanding why hiring happens without being picked up in our vacancy survey is a first step towards designing better measures of job-openings.
% and approximately 40 \% of all vacancies registered at the Public Employment Service do not have a counterpart in the survey data on vacancies.