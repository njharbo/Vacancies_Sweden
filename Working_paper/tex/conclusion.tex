\section{Conclusion}
\label{sec:conclusion}

This paper studied the relationship between vacancies and hires on the plant level using Swedish data. According to basic search-matching theory we should see no hiring without job-openings and we should expect the relationship between hiring and job-openings to be linear. When taking these predictions to the data we first need to translate the theoretical concept of a job-opening into an empirical counterpart. The preferred way to this in the empirical literature is via a survey based measure of vacancies. To the extent that this measure captures the concept of job-openings well, we should expect to see a tight relationship between vacancies and subsequent hires. I show that the relationship in the data is concave, rather than linear as predicted by the model, as one additional vacancy associated with less and less hiring. I also show the prediction of hiring on the plant level can be improved substantially by allowing plants' willingness to hire to depend not only on posted vacancies but also on plant size.

These observations motivates the formulation of an alternative measure of aggregate job-openings. Instead of using the sum of all vacancies, I use the the sum of all plants with a positive number of vacancies weighted by their size. This measure has the attractive feature of providing a better fitting aggregate matching function. The alternative measure also provides a new perspective on the outward shift in the Swedish Beveridge curve observed in the wake of the Great Recession. Some have suggested that this was caused by a deterioration of the matching efficiency. The findings in this paper however puts this hypothesis into doubt. Indeed, an alternative explanation is that the traditional vacancy measure has become a less good measure for the number of job-openings in the economy.

Overall, this paper suggests that more work is needed on how to best measure job-openings in the economy. Indeed, the finding that a substantial amount of hiring happens without vacancies in the preceding month, and that 37 \% of all vacancies registered at the PES do not have a counterpart in the survey data point to a reliability problem in our vacancy data. Better understanding the hiring that happen without vacancies is a first step towards addressing this problem.