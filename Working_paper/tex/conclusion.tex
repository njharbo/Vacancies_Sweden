\section{Conclusion}
\label{sec:conclusion}

In modern macroeconomic models job-openings are a key concept. To measure job-openings in the data the literature relies on vacancy data. Such data is either measured in surveys or in register data. Yet so far we know very little about how job-vacancies relate to actual hiring on the micro level. Insofar that job vacancies capture the concept of job-openings well, we should expect to see a tight relationship between vacancies and subsequent hires.
 
This paper is one of the first to study this relationship on the plant level. It does so using a novel Swedish dataset. Using this dataset, I show that the relationship between job-vacancies and subsequent hires is weak and concave. That is, one additional vacancy on the plant level predicts less and less hiring. I also show that the number of hires per vacancies (the \emph{vacancy yield}) varies in the cross-section of plants. In particular it is increasing in plant size. Moreover, plants hire many more workers than they post vacancies. I take these findings as evidence of vacancies being a poor measure of actual job-openings in the economy. 

I also show that it is possible to construct a better measure of job-openings. By allowing job-openings to depend not only on vacancies, but also on plant size, I show that it is possible to improve the ability to predict hiring on the plant-level by up to 40 \%. 

Building on these findings from the plant-level, I then construct an alternative measure of job-openings in the aggregate. Motivated by the concave relationship between vacancies and hires, and the fact that the vacancy yield is increasing in plant size, I propose \emph{the sum of plants with a positive number of vacancies weighted by size} as an alternative measure of job-openings in the economy. Interestingly, this measure also yields a better fitting aggregate matching function than the traditional vacancy measure.

This alternative measure for job-openings provides a new perspective on the apparent outward shift in the Beveridge curve after the Great Recession. Using the alternative measure for job-openings to analyze the recent developments on the Swedish labor market, the labor market appears less tight after the Great Recession and the outwards shift in the Beveridge curve was not as pronounced. \emph{Ceteris paribus} this lends less support to the hypothesis that matching efficiency has decreased after the crisis. 

Overall, my findings suggest that more work is needed on how to best measure job-openings in the economy. A substantial amount of hiring happens without preceding vacancies, and approximately 40 \% of all vacancies registered at the Public Employment Service do not have a counterpart in the survey data on vacancies. This points to a reliability problem in our vacancy data. Understanding why hiring happens without being picked up in our vacancy survey is a first step towards designing better measures of job-openings.
