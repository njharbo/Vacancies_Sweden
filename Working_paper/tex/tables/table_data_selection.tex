\begin{table}[htbp]\centering
\begin{threeparttable}
\def\sym#1{\ifmmode^{#1}\else\(^{#1}\)\fi}
\caption{Data selection}
\label{tab:data_selection}
\begin{tabularx}{\textwidth}{lXcc}
\hline\hline
                &&\multicolumn{1}{c}{Tax hires}&\multicolumn{1}{c}{Survey hires} \\
\hline

\hline

All && 1,006,525 &  428,016\\
- Non zero observations && 121,836 & 119,331  \\
With all background variables && 770,481 & 334,926 \\
- Non zero observations &&86,009 & 100,806 \\

\hline\hline
\end{tabularx}

\begin{tablenotes}
\item \footnotesize{ \emph{Notes:} The table shows the number of observations for each data-selection using tax- and survey-hires, respectively. Vacancies are matched with hires in next month. Non zero observations counts the number of observations in the given data selection where both hires and vacancies are non-zero.}
\item \footnotesize{ \emph{Source:} Statistics Sweden and IFAU. }
\end{tablenotes}
\end{threeparttable}
\end{table}
