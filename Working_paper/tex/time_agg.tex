\section{Dealing with time-aggregation}
\label{sec:time_agg}

An issue I have alluded to, but not dealt with so far, is that of time-aggregation. In section \ref{sec:basic_rel}, I associated 
 hiring in period $t$ with the number of vacancies posted in the middle of period $t-1$. This approach could be problematic for two reasons. First, a vacancy posted in the middle of month $t-1$ might be filled before the beginning of month $t$. Second, a hire made in period $t$ might be associated with a vacancy which was created after vacancies were counted in the middle of month $t-1$. 
 
To address this problem, I take the approach developed by \cite{Davis2013}. They set up a simple model, which captures the daily dynamics of vacancies and hires. Using a calibrated version of this model it is possible to compute (1) the number of vacancies in the end of month $t-1$, and (2) the number of hires in month $t$ associated with newly created vacancies in period $t$. 

Specifically, \cite{Davis2013} model the daily dynamics of of hires and vacancies using the following system of equations.
\begin{align}
h_{s,t}&=f_t v_{s-1,t} \label{eq:hires_dyn} \\
v_{s,t}&=(1-f_t)(1-\delta_t)v_{s-1,t} + \theta_t \label{eq:vacancies_dyn} 
\end{align}
Here $h_{s,t}$ is the number of hires at day $s$ in month $t$, $v_{s,t}$ is the number of vacancies at day $s$ in month $t$, $f_t$ is the daily job-filling rate, $\delta_t$ is the daily depletion rate of vacancies\footnote{That is, the daily rate by which vacancies are taken of the market without having been filled.} and $\theta_t$ is the inflow of new vacancies each day. Both $f_t$, $\delta_t$ and $\theta_t$ are assumed to be constant throughout each month. Equation \eqref{eq:hires_dyn} is thus telling us that the number of hires at day $s$ in month $t$ is equal to the number of vacancies yesterday multiplied by the vacancy filling rate. Likewise, equation \eqref{eq:vacancies_dyn} tells us that the number of vacancies at day $s$ in  month $t$ is equal to the number of vacancies from yesterday, which were not filled nor depleted, plus the inflow of new vacancies. 

Solving \eqref{eq:hires_dyn} and \eqref{eq:vacancies_dyn} forward yields an expression for stock of vacancies in the begining of month $t$ and the flow of hires during month $t$.
{\scriptsize\begin{align}
v_t&=\left( 1-f_t-\delta_t+\delta_t f_t \right)^\tau v_{t-1} + \theta_t \sum_{s=1}^{\tau} \left( 1-f_t-\delta_t+\delta f_t \right)^{s-1}  \label{eq:v_t} \\
h_t&=f_t v_{t-1} \sum_{s=1}^{\tau} \left( 1-f_t-\delta_t+\delta_t f_t \right)^{s-1} + f_t \theta_t \sum_{s=1}^{\tau} \left( \tau- s\right) \left( 1-f_t-\delta_t+\delta_t f_t \right)^{s-1} \label{eq:h_t}
\end{align}}
Here $\tau$ is the number of days per months. The first expression on the right-handside of equation \eqref{eq:v_t} denotes the number of vacancies from month $t-1$ that carried over to month $t$. The second expression captures the remainder from the flow of newly created vacancies. Likewise, the first right-handside expression in equation \eqref{eq:h_t} denotes hires out of vacancies posted in period $t-1$, while the second expression denotes hires out of newly created vacancies. 

Given $\tau$ and time series for the triplet $\{\delta_t, h_t, v_t \}$ one can solve this equation system numerically for the time series of $\{ f_t, \theta_t \}$. I set $\tau$ equal to 22. To calibrate $\delta$ I set $\tau \delta_t$ equal to monthly job-destruction rate.\footnote{Specifically, I set $\tau \delta_t$ equal to the average monthly probability of not staying in a regular contract. This data is available from the Swedish labor market survey. The approach is inspired by \cite{Davis2013} who do the same, although they allow $delta$ to be be time-varying. However, they also argue that the computation is insensitive to this parameter.} $h_t$ and $v_t$ are available from the \emph{Vacancy Survey} and the \emph{Short-Term Employment Statistics}. However, to use these numbers I first need to discuss the timing of measurement. As explained in Section \ref{sec:data} the observed number of hires in a given month is the sum of all hires throughout that month, while the observed number of vacancies is the stock in the middle of the month (Figure \ref{fig:timing_illustration}). To ensure timing consistency I thus need to construct the total number of hires in the interval between $V_t$ and $V_{t+1}$. To approximate this, I take the average number of hires between month $t$ and $t+1$. 

Initially, I use this method to compute the aggregate time-series for $f_t$ and $\theta_t$.  To do this, I rely on the published data series for aggregate number of hires and vacancies in the private sector (Figure \ref{fig:survey_openings_timeserie} and \ref{fig:hires_survey_timeserie}). These data-series are published only on quarterly frequency, why the resulting time-series will be averages over each quarter. Moreover, this frequency precludes me from approximating the number of hires between middle of each month. Instead, I have to assume that the aggregate flow of hires during a month approximates well the flow of hires from the middle of each month to the next. Further, a data-break exist in the aggregate serie for public sector hires in 2006. Consequently, I do the computation using private sector data only.

The resulting time-serie for aggregate job-filling rates is show in Figure \ref{fig:aggregate_filling_rates}. The calibrated daily filling rate varies in the interval $0.015-0.035 $, and have an average of 0.021 which corresponds to an average vacancy duration of 1/0.0216= 46days. This duration is relatively high compared to the duration of vacancies reported at the Public Employment Service (Figure \ref{fig:PES_duration}). Several reasons can be behind this discrepancy. First, the selection of vacancies differs across the two selections of data. Not all vacancies are reported at the Public Employment Service, and in the computation in \ref{fig:aggregate_filling_rates} only the private sector is included. Second, the duration measure itself is also different. For vacancies in the Public Employment Service we measure the publication period, whereas the computation in this section aims at measuring the period from vacancy creation to fill date. 

The time-serie for aggregate vacancy creations is shown in Figure \ref{fig:aggregate_creation_rates}. The monthly inflow of new vacancies varies in the interval $0.6-1.1 \%$ of employment with a mean of $0.81 \%$. The time-series is roughly stationary, but contains a negative blip around the financial crisis in 2008/09.

In Table \ref{tab:filling_creation_rates_by_industry} I report the job-filling and vacancy creation rates calibrated by industry. These numbers are calibrated using the unweighted micro data for hires and vacancies during the period 2001-2012. Across industries the daily job-filling rate varies between an average of 0.06 in farming  to 0.01 in transportation, mail and telecom. The monthly inflow of new vacancies varies from 3.9 \% of employment in Manufactoring to 0.4 \% of employment in Public and Personal Services. 

Using these calibrated numbers, I now attempt to correct the analysis done in Section \ref{sec:basic_rel} for the problem of time-aggregation. Specifically, I use the calibrated time-series for the relevant industry to predict for each plant (i) the number vacancies in the end of each month and (ii) the number of hires in the following month corrected for hires out of newly created vacancies. I do this by means of the two following formulas. 
\begin{align}
v_{t,ultimo}&=\left( 1-f_t-\delta_t+\delta_t f_t \right)^{\tau/2} v_{t,medio} + \theta_t \sum_{s=1}^{\tau/2} \left( 1-f_t-\delta_t f_t \right)^{s-1} \label{eq:vt_ultimo} \\
h_{t,corrected}&=h_t-f_t \theta_t \sum_{s=1}^{\tau} \left( \tau- s\right) \left( 1-f_t-\delta_t+\delta_t f_t \right)^{s-1} \label{eq:ht_corr}
\end{align}
In \eqref{eq:vt_ultimo}, the first expression on the right-hand side denotes the predicted number of remaining vacancies from the stock observed in the middle of the month. The second expression denotes the stock of remaining vacancies that were created after the middle of the month. In \eqref{eq:ht_corr} the first term is simply the stock of observed hires in the given month, while the second term expresses the expected number of hires made out of vacancies created during the relevant month.

Having computed $v_{t,ultimo}$ and $h_{t,corrected}$ I now redo the analysis from Section \ref{sec:basic_rel}. In Table \ref{tab:robust_timeagg_ols}, I re-estimate the relationship between hiring and vacancies using OLS while gradually increasing the number of plant- and firm-level characteristics. In Table \ref{tab:robust_timeagg_nls} I do the same, but with non-linear least squares. The pattern from Section \ref{sec:basic_rel} remains: (1) the estimated relationship between hires and vacancies is concave, not linear, (2) the exponent on vacancies goes towards zero as I increase the number of plant- and firm-level characteristics and (3) the fit of regression is improved by allowing job-openings to be a function of plant-size as well as of posted vacancies. 

Having accounted for time-aggregation, I also revisit the issue of hiring without preceding vacancies. In Figure \ref{fig:share_without}, I showed that the share of hires without vacancies in the preceding month varied in the interval 40-60 \%. In Figure \ref{fig:share_without_timeagg}, I redo this exercise having accounted for time-aggregation. Here the share of hiring without vacancies in the preceding month varies in the interval 20 - 30 \% for tax hires and 10-20 for survey hires. This suggests that time-aggregation can account for some of the observed hiring without preceding vacancies.

\subsection{Additional robustness checks}

In addition, I conduct three other robustness checks. 

First, one might question the choice of relating vacancies in one month to hires in the next month only. Especially, when contrasting the evidence on vacancy durations from the Public Employment Service from Figure \ref{fig:PES_duration} with the calibrated fill rates in Figure \ref{fig:aggregate_filling_rates}. To address the sensitivity of my analysis with respect to this choice, I redo the analysis in Table \ref{tab:main_ols} and \ref{tab:main_nls} while relating the number of vacancies in a given month the \emph{average number of hires over the two next months}. This analysis is presented in Table \ref{tab:robust:2m}, and does not overturn the results from Table \ref{tab:main_ols} and \ref{tab:main_nls}.

Second, a shortcoming of my dataset is the lack of a complete panel structure for vacancies. This makes it impossible to aggregate both vacancies and hires to the annual level, and then relate the yearly number of hiring to the yearly number of vacancies. However, for some plants I do have more than one observation per year. As a robustness check I therefore restrict my sample to plants, where I have at least three observations per year. Using this sample I relate the average number of hires per year to the average number of vacancies per year. The results from the exercise is also presented in Table \ref{tab:robust:average_year}, and do also not overturn the main results.

Third, one might be concerned that the results are driven by data selection, rather than explanatory power from the firm and plant level characteristics. Indeed, the number of observations drop as more variables are included Table \ref{tab:main_ols} and \ref{tab:main_nls}. Hence, one concern is that the better fit is not driven by the inclusion of  plant and firm characteristics, but instead the drop in number of observations. To ensure this is not the case, I redo the estimations where the sample is restricted to the subset where all variables are available (Table \ref{tab:plantlevel_rel:constant_sample}). This does not alter my results. 