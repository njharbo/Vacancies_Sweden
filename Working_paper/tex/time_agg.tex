\section{Dealing with time-aggregation}
\label{sec:time_agg}

A problem with the analysis so far is that it does not take the issue of time-aggregation into account. Indeed, above I associate hiring in month $t+1$ with number of vacancies posted medio month $t$. This may be problematic for both reasons. First, vacancies could be filled before the start of the next month. Second, some of the hiring done in next month may be associated with vacancies that were created after vacancies were counted in month $t$.

To address this problem, I take the approach developed by \cite{Davis2013}. They set up a model for the daily dynamics of vacancies, and using a calibrated version of this model one can compute (1) the number of vacancies in the end of each period and (2) the number of hires next period that are made via the filling of newly created vacancies. In particular, the daily dynamics of hires and vacancies is modeled using the following system of equations.
\begin{align}
h_{s,t}&=f_t v_{s-1,t} \label{eq:hires_dyn} \\
v_{s,t}&=(1-f_t)(1-\delta_t)v_{s-1,t} + \theta_t \label{eq:vacancies_dyn} 
\end{align}
here $h_{s,t}$ is the number of hires at day $s$ in month $t$, $v_{s,t}$ is the number of vacancies at day $s$ in month $t$, $f_t$ is the daily job-filling rate, $\delta_t$ is the daily and $\theta_t$ is the inflow of new vacancies each day. Both $f_t$, $\delta_t$ and $\theta_t$ are assumed to be constant throughout each month. 

Solving \ref{eq:hires_dyn} and \ref{eq:vacancies_dyn} forward yields the following expressions.
\begin{align}
v_t&=\left( 1-f_t-\delta_t+\delta_t f_t \right) v_{t-1} + \theta \sum_{s=1}^{\tau} \left( 1-f_t-\delta_t+\delta f_t \right)^{s-1} \\
h_t&=f_t v_{t-1} \sum_{s=1}^{\tau} \left( 1-f_t-\delta_t+\delta_t f_t \right)^{s-1} + f_t \theta_t \sum_{s=1}^{\tau} \left( \tau- s\right) \left( 1-f_t-\delta_t+\delta_t f_t \right)^{s-1}
\end{align}
where $\theta$ is the number of days per months. Given $\tau$ and time series for the triplet $\{\delta_t, h_t, v_t \}$ one can solve this equation system numerically the time series for $\{ f_t, \theta_t \}$. $h_t$ and $v_t$ is available from the data and I set $\tau=26$ (working days per month). $\delta_t$ is less obvious how to compute, but as of now I follow \cite{Davis2013} and $\tau \delta_t$ equal to monthly job-destruction rate.\footnote{Specifically, I set $\tau \delta_t$ equal to the monthly probability of not staying in a regular contract. This data is available the Swedish labor market survey.} TBD: However, as robustness I vary $\tau \delta_t$ in the interval $[0,10\%]$ and show that this impacts very little on the calibrated values for $f_t$ and $\theta_t$.

Figure \ref{fig:filling_rates} shows the calibrated time-series for $f_t$ and $\theta_t$. The calibrated monthly inflow of new vacancies is $0.6 \%$ of the labor force on average and varies in the interval $0.2-0.8 \%$. The daily fill-rate of vacancies has an average of $2.5 \%$, which corresponds to an average duration of 40days, and varies in the interval $1-3.5\%$. 

\begin{figure}[h]
\centering
%l b r t
\caption{Daily Job-Filling Rates and Flow of New Vacancies, 2001-2012}
\includegraphics[trim = 20mm 40mm 20mm 40mm, clip, width=\textwidth]{../../timagg/job_filling_creation_rate_figure}
\flushleft
\footnotesize{\emph{Source:} Own calculations on data from Statistics Sweden.}
\label{fig:filling_rates}
\end{figure}

Using these calibrated values I now attempt to address the time-aggregation problem. Specifically, I use the calibrated job-filling  and vacancy creation rates to compute (1) the predicted number of vacancies at each plant in the end of each month and (2) the number of hires in the following month corrected for the predicted hires caused by filling of newly created vacancies. According to the model the predicted number of vacancies in the end of the month can be written
\begin{align}
v_{t,ultimo}=\left( 1-f_t-\delta_t+\delta_t f_t \right)^{\tau/2} v_{t,medio} + \theta \sum_{s=1}^{\tau/2} \left( 1-f_t-\delta_t f_t \right)^{s-1}
\end{align}
whereas the number of hires corrected for hires associated with newly created vacancies in a given month reads
\begin{align}
h_{t,corr}=h_t-f_t \theta_t \sum_{s=1}^{\tau} \left( \tau- s\right) \left( 1-f_t-\delta_t+\delta_t f_t \right)^{s-1}
%&=f_t v_{t-1} \sum_{s=1}^{\tau} \left( 1-f_t-\delta_t+\delta_t f_t \right)^{s-1}
\end{align}
To compute $v_{t,ultimo}$ and $h_{t,corr}$ on the plant-level, I use values for $f_t$ calibrated on the industry level and to compute a plant-specific value for $\theta_t$ I weight the $\theta_t$ computed on the industry-level with the plant's employment share in the industry.

Having computed $v_{t,ultimo}$ and $h_{t,corr}$ I now redo the analysis from above. In Table \ref{tab:robust_timeagg} in the Appendix I re-estimate the relationship between hiring and vacancies and gradually increase the number of plant and firm -level characteristics. The pattern is unchanged: The relationship between hires and vacancies is concave, not linear, and the coefficient on vacancies goes towards zero as I increase the number of plant-level characteristics. 

\subsubsection{Additional robustness checks}

I also conduct a couple of additional robustness checks. First,  I relate the number of vacancies in a given month with the \emph{average of hires hires over the next two months} (Table \ref{tab:robust_2m_avg}). Second, I restrict my sample to plants where I have at least three observations for a given year. Using this sample I relate the average number hires with the average number of vacancies (Table  \ref{tab:robust_year_avg}). None of these robustness checks overturn the results.