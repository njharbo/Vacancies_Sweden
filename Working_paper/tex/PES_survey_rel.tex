\section{Survey and register based vacancy data}
\label{sec:PES_survey_rel} 

In this section, I document the firm level relationship between vacancies as measured in (i) the survey from Statistics Sweden and (ii) the register data from the Public Employment Service. 

Figure \ref{fig:agg_PES_survey_share} shows a cross-plot between these two measures on the firm level. If the two measures were identical, we should see the observations line up on the 45-degree line. Evidently, this is not the case. Instead, regressing the two measures on each other yields a $R^2$ of only 0.06. Moreover, 47 \% of the vacancies registered in the survey are not registered in the Public Employment Service. This is unsurprising since all firms can not be expected to use the Public Employment Service as the channel for posting vacancies. What is, however, more surprising is that 37 \% of the vacancies registered at the Public Employment Service are not counted in the survey.

\begin{figure}[h]
\centering
%l b r t
\caption{Vacancies registered in the survey and at the Public Employment Service 2004-2012 }
\includegraphics[trim = 0mm 0mm 0mm 0mm, clip, width=\textwidth]{../../Data/Server/figures/plot_PES_v_survey_all_cfarshare1.png}
\flushleft
\footnotesize{\emph{Notes:} The sample is restricted to firms with only one plant. The solid line depicts the 45-degree line.} \\
\footnotesize{\emph{Source:} Own calculations on data from Statistics Sweden and Swedish Public Employment Service.}
\label{fig:agg_PES_survey_share}
\end{figure}

Table \ref{tab:PES_survey} documents the heterogeneity in the ratio of vacancies registered the Public Employment Service to the number counted in the survey. Overall, this table suggests at the use of the Public Employment Service as recruitment channel varies substantially across firm characteristics. The share of PES to survey vacancies is largest within \emph{Public and personal services}, where the share is 79 \%, and smallest within \textit{Construction}, where the share is only 18 \%.  Across plant size there is not a clear pattern, while the use of the Public Employment Service is largest in the middle of the distribution of turnover and value-added.

      \begin{table}[htbp]\centering
      \caption{\label{tab:PES_survey} Survey and Public Employment Service vacancies on the firm level, 2004-2012}
			\begin{tabularx} {\textwidth} { l X c  c  c c c} \\ \hline %**{4}{R}
      
			\textbf{ } & &\scriptsize{PES share}  & \scriptsize{Average PES}  &  \scriptsize{Average Survey} &  \scriptsize{Share of PES} &  \scriptsize{Share of survey }  \\
			
			      \textbf{ } && \scriptsize{($\%$)}  & \scriptsize{($\#$)}  &  \scriptsize{($\#$)} &  \scriptsize{($\%$)} &  \scriptsize{($\%$)}  \\
			
      \midrule
			   && \multicolumn{5}{c}{By industry} \\
			\midrule
\input{../../Data/Server/tables/table_PES_survey_ratio_by_industry.txt}
     \midrule
						   && \multicolumn{5}{c}{By number of employees (deciles)} \\
			\midrule
\input{../../Data/Server/tables/table_PES_survey_ratio_by_size.txt}
     \midrule
						   && \multicolumn{5}{c}{By turnover (deciles)} \\
			\midrule
\input{../../Data/Server/tables/table_PES_survey_ratio_by_turnover.txt}
				\hline
\multicolumn{7}{l}{\footnotesize{\emph{Notes:} PES-share is the average number of PES vacancies divided by the average number survey   }} \\
 \multicolumn{7}{l}{\footnotesize{vacancies for each category. Share of PES / share of survey denotes how large a share of total  }} \\
 \multicolumn{7}{l}{\footnotesize{vacancies the category accounts for. (1): Trade, hotel and restaurants, (2): Transportation, mail and}} \\
 \multicolumn{7}{l}{\footnotesize{telecommunications, (3): Finance and business services, (4): Public and private services. }}   \\
\multicolumn{7}{l}{\footnotesize{\emph{Source:} Own calculations on data from Statistics Sweden and Swedish Public Employment Service.}} 
      \end{tabularx}
      \end{table}
			

\begin{table}[htbp]\centering
\caption{\label{tab:PES_survey} Survey and Public Employment Service vacancies on the firm level, 2004-2012}
\begin{tabularx} {\textwidth} { l X c  c  c c c} \\ \hline %**{4}{R}
\textbf{ } & &\scriptsize{PES share}  & \scriptsize{Average PES}  &  \scriptsize{Average Survey} &  \scriptsize{Share of PES} &  \scriptsize{Share of survey }  \\
\textbf{ } && \scriptsize{($\%$)}  & \scriptsize{($\#$)}  &  \scriptsize{($\#$)} &  \scriptsize{($\%$)} &  \scriptsize{($\%$)}  \\
\midrule
           && \multicolumn{5}{c}{By valueadded (deciles)} \\
\midrule
\input{../../Data/Server/tables/table_PES_survey_ratio_by_valueadded.txt}
\midrule
\hline
\multicolumn{7}{l}{\footnotesize{\emph{Notes:} PES-share is the average number of PES vacancies divided by the average number survey   }} \\
 \multicolumn{7}{l}{\footnotesize{vacancies for each category. Share of PES / share of survey denotes how large a share of total  }} \\
 \multicolumn{7}{l}{\footnotesize{vacancies the category accounts for.}} \\
\multicolumn{7}{l}{\footnotesize{\emph{Source:} Own calculations on data from Statistics Sweden and Swedish Public Employment Service.}} 
\end{tabularx}
\end{table}
			
			


Finally, Figure \ref{fig:agg_PES_survey_share} documents the time-serie for the aggregate share of vacancies registered at the Public Employment Service compared to the number counted in the survey data. During the interval 2004-2012 the share has varied in the interval $28-47$ \% and has displayed a increasing trend.

\begin{figure}[h]
\centering
%l b r t
\caption{Ratio Public Employment Service to survey vacancies, 2004-2012 }
\includegraphics[trim = 0mm 0mm 0mm 0mm, clip, scale=1]{../../Data/Server/figures/plot_PES_survey_ratio_incl_pub_2004_MA.pdf}
\flushleft
\footnotesize{\emph{Notes:} The figure shows a 12-month moving average of the aggregate ratio of PES to survey vacancies in the sample.} \\
\footnotesize{\emph{Source:} Own calculation on data from Statistics Sweden and Swedish Public Employment Service.}
\label{fig:agg_PES_survey_share}
\end{figure}