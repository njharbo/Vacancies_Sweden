In modern macroeconomic models job openings are a key component. Thus, when taking these model to the data we need an empirical counterpart to the theoretical concept of job openings. To achieve this, the literature relies on job vacancies measured either in survey or register data. Insofar, that this concept captures the concept of job-openings well we should see a tight relationship between vacancies and subsequent hires on the micro level. To investigate this, I construct a new dataset on Swedish hires and job vacancies on the plant level covering the period 2001-2012. I show that vacancies contain little power in predicting hires above (i) whether the number of vacancies is positive and (ii) plant size. Building on these findings, I propose an alternative measure of job openings in the economy. This measure has the attractive features of (i) better predicting hiring on the plant level and (ii) providing a better fitting aggregate matching function \emph{vis-a-vis} the traditional vacancy measure. Using the new measure, it is less clear that the Beveridge curve has shifted outwards in the aftermath the Great Recession.

