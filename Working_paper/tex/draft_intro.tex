\section{Introduction}

Job-openings a key concept in the search-matching literature. 

Thus, when taking the model to the data mapping the theoretical concept of a job-opening to an emperical counterpart is key. 
In practice survey data on vacancies is used.
And models predictions on the aggregate level are consistent with the behaviour of vacancies - at least quantitively.  

The consistency between data and model should however also be evaluated on its performance on the plant level. Model predicts hiring on firm level is function of aggregate tightness and number of job-openings in the firm. 

This paper makes progress on this. Using new data set .... I show that relationship is weak and non-linear, with little role for additional vacancies after the first.

One potential explanation behind this weak relationship is measurement problems. Indeed, vacancies in the data might not fully capture the relevant variable: firms willingness to hire. 

To investigate this, I allow firms willingness to hire not only to depend on vacancies in the data but also on other firm characteristics. Dramatically improves fit. Suggests size contains information on willingness to hire over and above what is included in the number of vacancies in the data. 

My plant-level regressions motivates/suggests a new measure for firms willingness to hire: number of recruiting plants weighted by their size. Explain.

I show that this crude measure have some interesting implications on the aggregate level. Fit. BC

Literature. 