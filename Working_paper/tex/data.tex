\section{Data}
\label{sec:data}

\subsubsection{Job vacancies}

For micro-data on job-vacancies, I draw the \textit{Swedish Job Vacancy Survey}.

The \textit{Swedish Job Vacancy Survey} is administered by Statistics Sweden and has been collected on a quarterly basis since 2001. Two vacancy concepts are available from this survey: (1) The number of available positions in each plant, which has been made \enquote{available for external job-seekers via the newspapers, internet or another mean of dissemination(\emph{spridningsform})}\footnote{In Swedish: \emph{Lediga jobb}.}, and (2) the number of these positions that the employer wishes to fill immediately\footnote{In Swedish: \emph{Vakanser}.}. This way, the former concept is a super set of the latter. In my study below, I rely on (1) which is the the widest definition of a vacancy.
%Figure XX in the appendix shows the form that the employer is asked to fill out.

The \textit{Swedish Job Vacancy Survey} is collected at the plant level, and all respondents are asked to report the number of vacancies in the middle of the reference month.\footnote{Specifically, the respondents are asked to report the number of job openings on the Wednesday closest to the 15th of the reference month.} For the private sector the sampling is done on the plant level with approximately 16 700 work places sampled each period. For the public sector the sampling was also done on the plant level until 2006Q2. In 2006Q2 the sampling was changed to the organizational level and on this level 650 organizations are sampled each period. Units larger than 100 employees are asked to do the reporting for each month of the relevant quarter, whereas units with less than 100 employees only are asked to report in the reference month. Reporting happens either via letter or online. Non-respondents are reminded via email, letter or a phone call. Until 2004, reporting was voluntary and the share of non-reporting units was 40 \% in the public sector and 20 \% in the private sector. In 2004 reporting became mandatory and currently the share of non-reporting units is 11 \% in the private sector and 2 \% in the public sector.
%\cite{SCB2015}
%\footnote{TBD: Describe difference between workplaces and firms in the public sector}
%Uppgiftsskyldighet enligt lag 2001:99. Vite can tildommes.
%Non-reporters. Source: mail from Vedrana
%The survey further contains information on how many of the open positions that are currently manned and unmanned, respectively, and how many of the positions that are available immediately.

The aggregate number and the plant-level distribution of vacancies is reported in Figure \ref{fig:survey_openings_timeserie} and \ref{fig:survey_openings_hist}. 

The aggregate numbers of the two vacancy measures are reported in Figure \ref{fig:survey_openings_timeserie} and Figure \ref{fig:PES_openings_timeserie}, respectively. As expected the level of vacancies in the survey is generally above the level of vacancies reported at the Public Employment Service, with the aggregate number of vacancies at the Public Employment Service on average being equal to 85 \% of that in the survey. %\footnote{To compute this ratio I restrict attention to survey and PES vacancies in the sample of plants in the \textit{Swedish Job Vacancy Survey}. Further, I restrict attention to firms with only one plant, so as to overcome the problem that survey vacancies are collected at the plant level whereas PES vacancies only are available on the firm level.} %See figure fig_openings_ratio_MA_timeseries

The distribution of vacancies in the survey (plant level) and PES (firm level) is shown in Figure \ref{fig:survey_openings_hist} and \ref{fig:PES_openings_hist}. For the survey based vacancy measure (Figure \ref{fig:survey_openings_hist}) the mean number of openings is 2.2, the median is 0 and the 90th percentile is 4. 73.4 \% of all plants do not report any vacancies in a given month, and the existence of vacancies appears sticky as only 13 \% of plants with zero vacancies in a given month report vacancies in the following month. Likewise 16 \% of the plant reporting vacancy in a given month also report vacancies in the next. 
%For vacancies recorded in the Public Employment Service the mean number of vacancies is 1.5, the median is 0 and the 90th percentile is 1. 86 \% of all firms do not report any vacancies at the PES in a given month. Moreover, the existence of vacancies at the PES is also persistent, with only 10.4 \% of all firms with zero vacancies in given month report vacancies in the next. Conversely, 63.5 \% of all firms with vacancies at the PES in one month also report vacancies in the next.

\subsubsection{Hires}

For hires I also have access to two data sources: (i) a survey-based measure from Statistics Sweden, and (ii) a register-based measure from the Swedish tax registry. 

The survey-based measure of hires stems from the \emph{Short-Term Employment Statistics} which is compiled by Statistics Sweden. This data is collected in combination with the Job Vacancy Survey described above, and thus contains the same sample of plants. From this survey, I construct the total number of hires in a given month by adding up all reported new hires on temporary and permanent contracts. In addition to the number of hires in each month the survey also contains the number of workers employed at each plant.

The second measure for hires is register-based and stems from the Swedish tax authorities. Specifically, the Institute for Evaluation of Labour Market and Education Policy (IFAU) maintains a database containing the start and end month of all employment spells as reported to the Swedish tax authorities. Along with the spell length the database contains an identifier for person, firm and plant. From this data, I compute the number of monthly hires as the number of spells that starts in an plant in a given month. To discard repeated, or interrupted, spells I remove all spells where the individual has been employed in the same plant or firm during the last 12 months. 

The aggregate number of hires from these two data sources are reported in Figure \ref{fig:hires_survey_timeserie} and \ref{fig:hires_tax_timeserie}. In most months the two measures are closely related (See a comparison in Figure \ref{fig:hires_tax_survey_timeserie}), but the general exception is January where register-based measure always exceeds the survey based by a large margin. This is likely to be caused by mis-measurement in the former, as employers for simplicity may report some spells as lasting for entire years instead of the correct duration in months. Moreover, the number of hires in the register based data displays a downwards trend, which is not seen in the survey based measure.

The distribution of hires across plants in shown in Figure \ref{fig:hires_survey_hist} and \ref{fig:hires_tax_hist}. For survey based measure of hires the mean is 1.8, the median 0 and the 95 \% percentile 6. As was the case with vacancies most plants (64.55 \%) do not hire in a given month according to the survey based measure. For the tax based measure the mean is 1.5, the median 0 and the 95 \% percentile 9. According to this measure 71.31 \% of all plants to not hire in a given month.

\subsubsection{Plant and firm background variables}

From the survey data in the \emph{Short-Term Employment Statistics} and the register data in the \emph{Swedish Firm Register}, both administered by Statistics Sweden, I furthermore have access to background information the plant and firm level. In particular, this background information contains information on the number of employees and industry of each plant, while turnover and value-added is available on the firm level from the Firm Register.  I report a summary of these variables in Table \ref{tab:tax_survey_comp}.

\subsubsection{Data selection}
\label{sec:data:selection}
In my analysis below I relate the number of vacancies in a given plant to the number of subsequent hires made at the same plant. For this purpose I need to decide on which measure for vacancies and hires to use. 

For vacancies I will use the survey based measure from Statistics Sweden. The reason for this is two-fold. First, survey based vacancy measures are broadly considered more reliable than register based data from public employment centers due to the possible selection problems in the latter. Second, the survey based measure will allow me to conduct my analysis on the plant, rather than the firm, level. %I will however investigate the firm-level relationship between the two vacancy measure in Section XX.

For hires I need to choose between the survey and tax based measure. The tax based measure has the advantage of being available for the full universe of plants during all months, whereas the survey based measure only is available for a plant if the plant is in the sample in the given month. As I wish to relate vacancies to \emph{subsequent} hires this present a problem as only larger larger are surveyed for consecutive months. This point is illustrated in Table \ref{tab:tax_survey_comp}, where I compare the characteristics of the observations from the data set on survey vacancies, where I also have access to tax- and survey-based hires in the subsequent month, respectively. The table shows that the data set with survey hires contain larger plants both in terms of employees, turnover (firm level) and value added (firm level). The tax based measure further has the advantage of providing more observations, as shown in Table \ref{tab:data_selection}. However, this point is less important once we restrict attention to observations (i) where all background variable are available and (ii) where hires and vacancies are non-zero. The tax based measure however has the problem of an upward biased number of hires in January, and downwards biased number the rest of year, as well as the downwards trend over time which is not observed in the survey based data. Due to these pros- and cons of each measure of hires, I will below relate vacancies to subsequent hires using \emph{both} the survey and tax based measure of hires.

%Below I will work with two different selections of the data above. 

%In Section \ref{sec:basic_rel}, I relate the number of vacancies in the \textit{Job Vacancy Survey} to the number of subsequent hires on the plant level. The reason I use the survey data, rather than the register data, is that the former broadly is considered more reliable due to the selection problems in the register data from the Public Employment Service. For hires I rely on the register data from the tax authorities. However, to ensure data quality of the hiring data, I cross-check the tax register data with the survey-based measure of hires in the \emph{Short-Term Employment Statistics}. In particular, I restrict the sample to plants where the number of hires in the tax-register and survey data is the same for the month of vacancy measurement.\footnote{I do not impose this restriction in the subsequent month, in which the relevant measurement of hiring is made, as the survey-data for hiring only is available in the same months as vacancies are measured.} 

%In section \ref{sec:PES_survey_rel}, I relate the number of vacancies in the survey to the number vacancies registered at the Public Employment Service. This is straightforward for the public sector after 2006Q2, as collection of both measures is done on the firm level. However, for the private sector, and the public sector before 2006Q2 the comparison is challenged by the fact that the survey data for vacancies only is available on the plant level, while the vacanccy data from the Public Employment Service only is available on the firm level. I address this problem by restricting the dataset to the sub-sample where only one plant per firm exists. 
%One concern with this approach is that the sub-sample of of firms with only one plant is not representative. To address this concern I repeat the analysis in Section  \ref{xx} for all firms, but where I predict the number of survey vacancies on the firm level by assuming that the share of vacancies on at a given plant is proportional to the share of employees at this plant.
 
%I merge data for the two measures of openings on the firm level. This is straightforward for the public sector after 2006Q2, as collection of both measures is done on the firm level. However, for the private sector and the public sector before 2006Q2 the merging on the firm level is challenged by variation in the level of data collection. Indeed, in the survey data collection is done on the establishment level, while the data in the PES only is available on the firm level. Hence, I predict the number of survey vacancies on the firm level by assuming that the share of vacancies on at a given plant is proportional to the share of employees at this plant.\footnote{I can, however, check my results in section \ref{sec:openings} against the results in the sub-sample where only one establishment per firm exists. In this sub-sample the number of survey vacancies on the firm and establishment level will be the same.} 

%[Data sources] We use firm-level vacancy data from two different sources. First, for vacancies we use the database from Swedish Public Employment Service (PES). This database contains start and end dates on each listed vacancy at the PES. Each vacancy is registered on the firm level. Second, we use a survey-based measure for vacancies from the Job Vacancy Survey (\emph{konjunkturstatistik över vakanser}) collected by the Statistics Sweden. This is a monthly representative survey covering approximately 18 700 private work places on the plant level, and 650 public work places on the organizational level (juridisk enhet). Here a vacancy is defined as an availiable position, which on a given has been announced and been made avaliable for external job-seekers but not yet is filled. The announcement can have been made through news papers, internet or through personal contacts. Thus, the definition of vacancies in the survey is broader than the definition in the PES database.
