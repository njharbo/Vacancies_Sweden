\section{Introduction}

One of the puzzles in macroeconomics after the Great Recession has been why unemployment in a number of advanced countries has been high, while job-openings at the same time has appeared to be plenty. This observation is captured by the notion that the \emph{Beveridge curve} has shifted outwards in some OECD countries -- including Sweden. In this paper, I investigate whether problems in measuring job-openings can be part of the story in the case of Sweden. My starting point for this study is a plant-level hiring equation, which derives from the standard search and matching model. I estimate this using Swedish plant-level data for vacancies and hires. I find that the vacancy measure is only weakly related to hiring on the plant-level, and that the number of hires per vacancy (the \emph{vacancy yield}) varies across the distribution of plants. Based on these findings I construct an alternative measure of job-openings, which builds on the extensive margin of vacancies and plant size. This measure improves the fit of the plant-level hiring equation and the aggregate matching function by XX\% and XX\%, respectively. Also, when using this measure to analyze the Swedish labor market experience after the Great Recession, job-openings appear less plenty after the recession and the outwards shift in the Beveridge curve was less pronounced. 

Job-openings are a key concept in modern macroeconomic models. Within the search-matching framework, which is an important building block in modern macroeconomics, we need to know the number of job-openings to infer how tight the labor market is. And on the micro level a hire is made, when a job-opening and an unemployed worker are matched via the aggregate matching function.

Thus, when taking these models to the data we need to construct a mapping from the theoretical concept of a job-opening in to an empirical counterpart. To achieve this mapping the literature relies on data for job-vacancies. These are either measured via surveys, where employers are asked about how many jobs they are currently trying to fill, or via register data on job vacancies made in newspapers or employment centers. We use these measures to guide our discussion about the aggregate state of the labor market, and to evaluate the predictions of our models. Yet so far, we know very little about how these vacancy measures relate to actual hiring on the micro level. Insofar that job-vacancies capture the notion of job-openings well, we should expect to see a tight relationship between job-vacancies and subsequent hires on the micro level. 

Specifically, aggregate hires can according to the textbook search and matching model be written as $H=AV^{1-\alpha}U^{\alpha}$
where $A>0$, $0<\alpha<1$, $V$ vacancies and $U$ unemployment. Assuming heterogeneity across plants, this implies a hiring equation on the plant-level of $H_j=A\left(\frac{U}{V}\right)^{\alpha} V_j$. While the aggregate matching function often is estimated in order to gauge aggregate matching function, so far very little evidence on the plant-level function exists, as it requires data for both hires and vacancies on the micro-level.  

This paper is among the first to investigate this micro level relationship. Specifically, I study a Swedish dataset with hires and job-vacancies on the plant level. Using this data, I find that the relationship between job-vacancies and subsequent hiring is weak and concave, in contrast to linear as predicted by the standard search and matching model. That is, variation in vacancies explains little of the variation in hiring on the plant level, and additional vacancies predict less and less hiring. Moreover, plants hire many more workers than the vacancies we observe. I also show that it is possible to improve the fit of the plant-level regression by XXX\% (measured by the adjusted $R^2$), by allowing the measure of job-openings to depend not only on listed vacancies, but also on plant-size. 

Building on these findings, I propose an alternative measure for the aggregate number of job openings in the economy. Motivated by the concave relationship between vacancies and hires on the plant level, and the predictive power of plant size, I use the \emph{number of plants with a positive number of vacancies weighted by size} as an alternative measure of the total job openings. Using this measure improves the fit of the aggregate matching function by XXX\%.

These findings provides a new perspective on an ongoing policy discussion. As mentioned above an important policy discussion in the wake of the Great Recession has been why unemployment has been high in a number of OECD countries (including Sweden) in spite of the stock of vacancies also being high. Some economists and policymakers have argued that declining match efficiency is behind this outwards shift in the Beveridge curve \citep{Hall2015, Riksbank2012}. My findings provides a new perspective on this discussion. Using the alternative measure of job-openings developed in this paper, the Swedish labor market appears less tight after the Great Recession. The reason is that vacancies have rebounded less in plants where the vacancy yield is high. Consequently, after the Great Recession the traditional vacancy measure may have overstated the number of job-openings in the economy and made the labor market appear too tight. Thus, my finding provides one potential explanation behind why the job-finding probability after the Great Recession is lower than what a matching function estimated on historical data predicts. Specifically, the predicted job-finding probability during 2008-2013 is on average 2.2 percentage points lower than when using the traditional vacancy measure. This constitutes 24\% of the post Great Recession breakdown in the relationship between hires and labor market tightness. 

My plant-level findings could however also have an alternative interpretation. Specifically, the varying number of hires per vacancy across the size distribution of plants could be interpreted as a varying \emph{match efficiency}. To investigate the implication of this interpretation, I build a simple matching model with heterogeneity in the match efficiency across plant-size.\footnote{This model is inspired by \cite{Kroft2016}, who also allow for heterogeneity in match efficiency across the distribution of unemployment duration.} Here a shift in the distribution of vacancies towards plants with lower match efficiency will \emph{per se} lower the job-finding probability. In this model a part of the lower job finding probability after the Great Recession can be explained by a shift in the distribution of vacancies towards smaller plants with lower match efficiency. However, the contribution is small, as it only can account for a 0.2 [TBD: CHECK] percentage point drop in the job-finding probability during 2008-2013 - thus accounting for less than XX\% of the total post-recession drop.

\subsubsection{Related literature}

My study relates to at least four strands of literature. 

First, and closest related, is the paper by \cite{Davis2013}. They analyze the relationship between hires and a survey-based measure of vacancies (JOLTS) on the plant level in the United States. Doing so they document how hires per vacancy, the \emph{vacancy yield}, behaves in the cross-section of plants and across time. Moreover, they develop a generalized matching function with a role for \emph{recruitment effort}. They show that variation in recruitment effort can partly explain the recent break-down in the matching function for the US. This paper takes a somewhat different approach. Instead of introducing a time-varying measure of recruitment intensity, I construct an alternative measure of job-openings which builds both on vacancies and plant-characteristics. %As I will argue below, this measure has the advantages of (i) better predicting hires on the plant-level and (ii) yielding a better fitting matching function on the aggregate level. %In addition, my paper also makes a contribution by documenting how survey and register-based measures of vacancies relate to each other on the micro level. 

Second, there exists a vast literature which estimates matching functions using the aggregate number of vacancies, unemployment and job-finding probabilities. A review of this literature is available in \cite{Pissarides2000}, but some key papers include \cite{Blanchard1990, Berman1997, Yashiv2000, Hansen2004, Sunde2007, Gross1997, Entorf1998, Feve1996}. My paper adds to this literature by discussing the micro-level properties of the vacancy data that goes into the estimation. 

Third, another strand of literature discusses the duration of vacancies on the firm level, and how this duration is determined \citep{Ours1991, Burdett1998, Barron1997, Holzer1990}. Here vacancies are studied on the micro level, but in isolation. My paper adds to this literature by investigating the \emph{link} between vacancies and hires on the micro level. 

Fourth, my paper relates to the recent debate on the Beveridge curve movements after the Great Recession. As documented by \cite{Hobijn2012} the Beveridge curve has shifted outwards in a number of OECD countries in the aftermath of the Great Recession. Some, non-mutually excluding, hypothesizes have been put forward to explain this apparent puzzle. \cite{Hall2015} have argued that declining matching efficiency in the pre-crisis period is behind the outward shift in the Beveridge curve in the United States. \cite{Riksbank2012} has argued that a similar mechanism has been operating in Sweden. Another hypothesis has been put forward by \cite{Kroft2016}. They argue that duration dependence in workers' transition rates between employment, non-employment and non-participation can account for much of the outward shift in the Beveridge curve in the United States. Finally, \cite{Davis2013} have argued that variation in firms’ recruitment intensity can explain parts of the outward shift. I add to this literature by arguing that mis-measurement of job-openings can explain some of the outward shift in the case of Sweden. 

\subsubsection{Organization}

The paper proceeds as follows. In Section \ref{sec:data}, I describe my data sources and how the database is constructed. In Section \ref{sec:basic_rel}, I document the relationship between vacancies and hires on the plant level. However, to analyze this relationship properly one has to take the issue of time-aggregation into account. I do this in Section \ref{sec:time_agg}, and show that this does not overturn the basic findings. In Section \ref{sec:agg_implications}, I build on my findings from the previous two sections and propose a new measure of aggregate job-openings in the economy. Section \ref{sec:conclusion} concludes.