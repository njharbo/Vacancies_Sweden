\section{Introduction}

One of the puzzles in macroeconomics after the Great Recession has been why unemployment in a number of advanced countries has been high, while job-vacancies at the same time has appeared to be plenty. This observation is captured by the notion that the \emph{Beveridge curve} has shifted outwards in many OECD countries (Figure TBD). In this paper, I argue that measurement problems can be part of the story in the case of Sweden. Using a novel dataset, I show that our vacancy measure is poorly related to hiring on the plant-level, and that the number of hires per vacancy (the \emph{vacancy yield}) varies across plant characteristics. Moreover, I construct an alternative measure of job-openings, which builds on the extensive margin of vacancies and plant size. This measure outperforms the traditional vacancy measure in predicting hiring on the plant level, and interestingly it also yields a better fitting matching function on the aggregate level. When using this measure to analyze the Swedish labor market after the Great Recession, the outward shift in the Beveridge curve is less pronounced. 

Job-openings are a key concept in modern macroeconomic models. Within the search-matching framework, which is a important building block in modern macroeconomics, we need to know the number of job-openings to infer how tight the labor market is. And on the micro level a hire is made, when a job-opening and an unemployed worker are matched via the aggregate matching function.

Thus, when taking these models to the data we need to construct a mapping from the theoretical concept of a job-opening in to an empirical counterpart. To achieve this mapping the literature relies on data for job-vacancies. These are either measured via surveys, where employers are asked about how many jobs they are currently trying to fill, or via register data on job vacancies made in newspapers or employment centers. We use these measures to guide our discussion about the aggregate state of the labor market, and to evaluate the predictions of our models. Yet so far, we know very little about how these vacancy measures relate to actual hiring on the micro level. However, insofar that job-vacancies capture the notion of job-openings well, we should expect to see a tight relationship between job-vacancies and subsequent hires on the micro level. 

This paper is one of the first to investigate this link. Specifically, I construct a novel Swedish dataset with hires and job-vacancies on the plant level. Using this data, I show that the relationship between job-vacancies and subsequent hiring is weak and concave, in contrast to linear as predicted by the standard search and matching model. That is, additional vacancies on the plant level predicts less and less hiring. Moreover, plants hire many more workers than they post vacancies.  I take these findings as evidence of vacancies being a poor measure of actual job-openings. 

I also show that it is possible to construct a better measure of job-openings. Indeed, by allowing job-openings to depend not only on listed vacancies, but also on plant size, I show that it is possible to improve the capability to predict hiring on the plant level by approx. 40 \%. Building on these findings, I propose an alternative measure for the aggregate number of job openings in economy. Motivated by the concave relationship between vacancies and hires on the plant level, and the predictive power of plant size, I use the \emph{number of plants with a positive number of vacancies weighted by size} as an alternative measure of job openings in the economy. I show, that this measure also has the appealing feature of providing a better fitting matching function on the aggregate level. 

These findings potentially have  important policy implications. As mentioned above an important policy discussion in the wake of the Great Recession has been why unemployment has been high in a number of OECD countries (including Sweden) in spite of the stock of vacancies also being high. Figuring out why unemployment is high is first order, when designing policies to bring it down. If unemployment is high due to lack of demand, then more expansionary policies can be expected to bring it down. But the joint incidence of high unemployment and vacancies speaks against this hypothesis. This has let some economists and policymakers to argue that declining match efficiency, rather than demand deficiencies, is behind the high level of unemployment \citep{Hall2015, Riksbank2012}.

My findings provides a new perspective on this discussion. Using the alternative measure of job-openings developed in this paper, the Swedish labor market appears less tight after the Great Recession. Moreover, the outward shift in the Beveridge is less pronounced. The reason is that the rebound in vacancies after the Great Recession was driven by vacancy postings in smaller plants, where the vacancy yield is lower. Consequently, after the Great Recession the traditional vacancy measure may have overstatede the number of job-openings in the economy and made the outward shift in the Beveridge curve appear too large. \emph{Ceteris paribus} this reduces the support for the hypothesis that the higher level of unemployment in Sweden after the Great Recession was caused by a decline in matching efficiency on the labor market.

Finally, I also provide some new evidence on the relationship between two often used empirical measures of job-openings: (i) survey data compiled by statistical agencies and (ii) register data on vacancies registered in databases maintained by public job-centers. The former type is often preferred, as this avoids the selection problems that can be present in the register data. However, due to limited availability of the survey based measure \citep{Elsby2015} the data from public job-centers are still often used as proxy, \citep{Berman1997, Carlsson2013, Hansen2004,  Wall2002, Yashiv2000}. Hence, an important question, which has not been investigated so far, is how the two vacancy measures relate on the micro level. I show that the relationship between openings registered at the Swedish Public Employment Service and vacancies reported in the survey is weak on the firm level. Not surprisingly 47 \% percent of the vacancies reported in the survey do not have a counterpart in the database of the Public Employment Service. More surprisingly is that 37 \% of all vacancies reported in the database of the public employment service do not have a counterpart in the survey data. Across firms there is also heterogeneity in the use of the Public Employment Service, with the public sector and middle-sized firms having the largest share of openings int eh Public Employment Service. 

\subsubsection{Related literature}

My study relates to at least four strands of literature. 

First, there exists a vast literature which estimates matching functions using the aggregate number of vacancies, unemployment and job-finding rates. A review of this literature is available in \cite{Pissarides2000}, but some key papers include \cite{Blanchard1990, Berman1997, Yashiv2000, Hansen2004, Sunde2007, Gross1997, Entorf1998, Feve1996}. My paper adds to this literature by discussing the micro-level properties of the vacancy data that goes into the estimation. 

Second, another strand of literature discusses the duration of vacancies on the firm level, and how this duration is determined \citep{Ours1991, Burdett1998, Barron1997, Holzer1990}. Here vacancies are studied on the micro level, but in isolation. My paper adds to this literature by investigating by investigating the link between vacancies and hires on the micro level. 

Third, and closest related, is the paper by \cite{Davis2013}. They analyze the relationship between hires and a survey based measure of vacancies (JOLTS) on the plant level in the US. The document how hires per vacancy, the \emph{vacancy yield}, behaves in the cross- and time-section. Moreover, they construct a measure of \emph{recruitment intensity}, and show how variations in this partly explains in the recent breakdown of the matching function in the United States. My paper takes a different approach. Instead of introducing a time-varying measure of recruitment intensity, I construct an alternative measure of job-openings which builds both on vacancies and plant-characteristics. As I will argue below, this measure has the advantages of (i) better predicting hires on the plant-level and (ii) yielding a better fitting matching function on the aggregate level. In addition, my paper also makes a contribution by documenting how survey and register based measures of vacancies relate to each other on the micro level. 

Fourth, my paper relates to the recent debate on the Beveridge curve movements after the Great Recession. As documented by \cite{Hobijn2012} the Beveridge curve has shifted outwards in a number of OECD countries in the aftermath of the Great Recession. Some, non-mutually excluding, hypothesizes have been put forward to explain this apparent puzzle. \cite{Hall2015} have argued that declining matching efficiency in the pre-crisis period is behind the outward shift in the Beveridge curve in the United States. \cite{Riksbank2012} has argued that a similar mechanism has been operating in Sweden. Another hypothesis has been put forward by \cite{Kroft2016}. They argue that duration dependent transition rates between employment, non-employment and non-participation can account for much of the outward shift in the Beveridge curve in the United States. Finally, \cite{Davis2013} have argued that variation in firms’ recruitment intensity can explain parts of the outward shift. I add to this literature by arguing that mis-measurement of job-openings can explain some of the outward shift in the case of Sweden. 

\subsubsection{Organization}

The paper proceeds as follows. In section \ref{sec:data}, I describe my data sources and how the database is constructed. In section \ref{sec:basic_rel}, I document the relationship between vacancies and hires on the plant level. However, to analyze this relationship properly one has to take the issue of time-aggregation into account. I do this in section \ref{sec:time_agg}, and show that this does not overturn the basic findings. In Section \ref{sec:agg_implications}, I build on my findings from the previous two sections and propose a new measure of aggregate job-openings in the economy. In Section \ref{sec:PES_survey_rel}, I document the relationship between the survey and register based measure of vacancies on the firm level. Section \ref{sec:conclusion} concludes.