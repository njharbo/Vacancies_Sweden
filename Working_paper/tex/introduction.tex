
\section{Introduction}

The outward shift of the Beveridge curve in a number of OECD countries after the Great Recession constitutes a puzzle. In this paper, I argue that measurement problems can be part of the story in Sweden. Using a novel dataset, I show that vacancies are poorly related to subsequent hiring on the plant-level. Moreover, I show that the fit of the relationship can improved by accounting for plant size and the distribution of vacancies. Informed by this, I construct a new measure for job openings in the aggregate. According to this new measure, the shift in the Beveridge curve after the Great Reccesion is less clear. 

Job openings are a key component in labor market analysis within the search-matching framework. On the micro level a hire is made when a job-opening and an unemployed worker are matched, and exogenous shocks to the model propagate via their impact on firms' job-posting behavior.

Thus, to take the search-matching model to the data we require a mapping from the theoretical concept of a job-opening to an empirical counterpart. To achieve this, the literature relies on data for job vacancies. Either via survey data, where the sampled companies are asked to report how many jobs they currently are trying to fill, or via register data on job vacancies made in newspapers or public employment centers. Evaluating the model using such data shows that the model is able to reproduce some key features of the data - at least qualitatively. \footnote{There is however debate about whether the model is account quantitatively for the variations seen in the data. Specifically,\cite{Shimer2005} has however argued that the textbook model is unable to account for the observed business cycle fluctuations in unemployment and vacancies.} Little is however known about the relationship between vacancies and hires on the plant level. Insofar that the our empirical measures for vacancies capture job-openings well, we should expect to see a tight relationship between vacancies and subsequent hiring on the plant level. 

To assess this, I construct a new dataset with vacancies and hires on the plant level. Specifically, I construct a database with monthly hires and vacancies on the plant level in Sweden during the period 2001-2012. Data on hires is constructed using both data from Swedish tax-records and survey data from Statistics Sweden. Data on vacancies is compiled using survey data from Statistics Sweden and register data on vacancies from the Swedish Public Employment Service (PES). 

Using this data, I investigate the relationship between vacancies and hires on the plant level.  First, I show that the relationship between vacancies and hires on the plant level is weak and concave in contrast to linear as predicted by the standard model. That is, additional vacancies predicts less and less hiring. Second, I show that the fit of the hiring regression on the plant level can be improved by aprox. 40 \% by allowing plants' willingness to hire to depend not only on listed vacancies but also on plant size. 

Building on these findings, I propose an alternative measure for the aggregate number of job openings in economy. Motivated by the concave relationship between vacancies and hires, and the predictive power of plant size, I use the \emph{number of plants with a positive number of vacancies weighted by size} as an alternative measure of job openings in the economy. I show, that this measure has the appealing feature of providing a better fitting aggregate matching function. 

This alternative measure also provides a new perspective on the apparent outward shift in the Swedish Beveridge curve following the Great Recession. In the wake of the Great Recession Sweden, like other OECD countries, saw an outward shift in the Beveridge Curve. One interpretation of this shift is that matching quality has declined. Another interpretation is that the traditional vacancy measure has become a less good measure for aggregate job-openings. Indeed, using the alternative measure the shift in the Swedish Beveridge curve after the Great Recession becomes less clear.

Finally, I also provide some new evidence on the relationship between two often used empirical measures of job-openings: (i) survey data compiled by statistical agencies and (ii) register data on vacancies registered in databases maintained by public job-centers. The former type is preferable, as this avoids the selection problems that can be present in the latter data, as not all companies rely on public job-centers as their recruitment source. Due to limited availability of a survey based measure data from public job-centers are still often used as proxy, \citep{Berman1997, Carlsson2013, Hansen2004,  Wall2002, Yashiv2000}. However, there are two obvious concerns with the latter measure. First, postings at the job-centers may not be representative for all job-openings in the economy. Second, the propensity of firms to use the job-centers as a recruitment channel may vary over time.  Not accounting for these problems may lead to spurious conclusions about matching efficiency on the labor market. Although these concerns are well-known the size of the problem has not previously been documented on the firm level. I show that the relationship between openings registered at the Swedish Public Employment Service (PES) and vacancies reported in the survey is weak on the firm level. Not surprisingly 47 \% percent of the vacancies reported in the survey do not have a counterpart in the PES. More surprisingly is, however, that 37 \% of all vacancies reported in the PES do not have a counterpart in the survey. Across firms there is also substantial heterogeneity in the use of PES opening, with the public sector and firm in the middle of the turnover and valueadded distribution having the largest share of PES to survey openings. During 2004-2012 the aggregate share of PES openings to survey openings in the sample varies in the interval 28-47 \%.

\subsubsection{Related literature}

My study relates to at least three strands of literature. 

First, a vast literature estimates matching function using the aggregate number of vacancies, unemployment and job-finding rates \citep{Blanchard1990, Berman1997, Yashiv2000, Hansen2004, Sunde2007, Gross1997, Entorf1998, Feve1996}.\footnote{An overview of the literature is available in \cite{Pissarides2000}}. My paper adds to this literature by discussing the micro-level properties of the vacancy data that goes into the estimation. 

Second, a strand of literature discusses the duration of vacancies on the firm level, and how this duration is determined \citep{Ours1991, Burdett1998, Barron1997, Holzer1990}. Here vacancies are studied on the micro level, but in isolation. My paper adds to this literature by investigating by investigating the link between vacancies and hiring on the micro-level. 

Third, and closest in spirit, is the paper by \cite{Davis2013}. They analyze the relationship between hires and a survey based measure of vacancies (JOLTS) on the plant level in the US. The document how hires per vacancy, the \emph{vacancy yield}, behaves in the cross- and time-section. Moreover, they construct a \emph{recruitment intensity}, and show how variations in this partly explains in the recent breakdown of the matching function. My paper takes a different approach. Instead of introducing a recruitment intensity measure that is varied to vary over time, I construct an alternative measure of job-openings which builds both on vacancies and plant-characteristics. As I will argue below, this measure has the advantages of (i) predicting hires better on the plant-level and (ii) yielding a better fitting aggregate matching function. In addition, my paper also makes a contribution by documenting the relationship between survey and register based vacancy measures on the firm level. 

Fourth, my paper relates to the recent debate on the Beveridge curve movements. As documented by \cite{Hobijn2012} the Beveridge curve has shifted outwards in a number of OECD countries in the aftermath of the Great Recession. A number of, non-mutually excluding, hypothesizes have been put forward to explain this apparent puzzle. \cite{Hall2015} have argued that declining matching efficiency in the pre-crisis period is behind the outward shift in the Beveridge in the United States. \cite{Riksbank2012} has argued that a similar mechanism has been operating in Sweden. Another hypothesis has been put forward by \cite{Kroft2016}. They argue that (i) duration dependent transition rates between employment, non-employment and non-participation can account for much of the outward shift in the Beveridge curve in the United States. Finally, \cite{Davis2013} have argued that variation in firms’ recruitment intensity can explain parts of the outward shift. I add to this literature by arguing that mis-measurement of job-openings can explain much of the outward shift in the case of Sweden. 

\subsubsection{Organization}

The paper proceeds as follows. In section \ref{sec:data}, I describe the data sources and how the database is constructed. In section \ref{sec:basic_rel}, I document the relationship between vacancies and hires on the plant level. However, to analyze this relationship properly one has to take the issue of time-aggregation into account. I do this in section \ref{sec:time_agg}, and show that this does not overturn the basic findings. In Section \ref{sec:agg_implications}, I build on my findings from the previous two sections and propose a new measure of aggregate job-openings in the economy. In Section \ref{sec:PES_survey_rel}, I document the relationship between the survey and register based measure of vacancies on the firm level. Section \ref{sec:conclusion} concludes.